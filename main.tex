\documentclass[12pt]{article}

% ---------- Пакети ----------
\usepackage[bulgarian]{babel}
\usepackage[T1]{fontenc}
\usepackage[utf8]{inputenc}
\usepackage{amsmath, amssymb}
\usepackage{geometry}
\usepackage{tikz}
\usepackage{pgfplots}
\pgfplotsset{compat=1.18}

\geometry{margin=2.5cm}

\title{\textbf{Производна: наклон, допирателна и „скорост на промяна“}\\
Лекция 2 — подготовка за диференциално смятане и AI}
\author{}
\date{}

\begin{document}
\maketitle

% =================================================
\section*{0. Цел на лекцията}

Днес:
\begin{itemize}
  \item правим \textbf{кратък преговор} на функция, графика, наклон и идея за граница
  \item строим идеята за \textbf{производна} като \emph{наклон в точка} и \emph{моментна скорост}
  \item стигаме до \textbf{официалната дефиниция} (накрая на интуицията)
  \item извеждаме и/или доказваме \textbf{част от правилата} и правим \textbf{таблица с всички основни правила}
  \item решаваме \textbf{първите 20 задачи } с подробни решения
\end{itemize}

\subsection*{Въпроси за бърза проверка}
\begin{enumerate}
  \item Кое е \textbf{по-важното} в тази лекция: формулите или смисълът зад тях? (Обясни с 1 изречение.)
  \item Дай пример за „скорост на промяна“ от реалния живот (не математика/физика по учебник).
  \item Какво очакваш да може да прави производната, което наклонът на права вече прави лесно?
\end{enumerate}

% =================================================
\section*{1. Преговор (5--10 мин)}

\subsection*{1.1 Функция}
Функция е правило, което на всеки допустим вход \(x\) съпоставя точно една стойност \(f(x)\).
Графиката е множеството точки \((x,f(x))\).

\subsection*{Въпроси}
\begin{enumerate}
  \item Ако имаме две различни стойности \(y\) за един и същ \(x\), това функция ли е?
  \item Кое е графиката: формулата \(f(x)\) или множеството точки \((x,f(x))\)?
  \item Дай пример за правило, което \textbf{не} е функция (с думи, не е нужно формула).
\end{enumerate}

\subsection*{1.2 Наклон на права}
За права \(y=ax+b\) наклонът е \(a\). Това е:
\[
\text{наклон}=\frac{\Delta y}{\Delta x}=\frac{y_2-y_1}{x_2-x_1}.
\]
Интуиция: „колко се вдига \(y\), когато \(x\) се увеличи с 1“.

\subsection*{Въпроси}
\begin{enumerate}
  \item Какво означава наклон \(m=3\) с думи?
  \item Ако \(x_2-x_1<0\), може ли наклонът пак да е положителен? (Да/не + 1 изречение.)
  \item Намери наклона на правата през точките \((1,2)\) и \((3,8)\).
\end{enumerate}

\subsection*{Графика: две прави с различен наклон}
\begin{center}
\begin{tikzpicture}
\begin{axis}[
    axis lines=middle, grid=both,
    xmin=-3,xmax=3,ymin=-3,ymax=5,
    xlabel={$x$}, ylabel={$y$},
    legend style={at={(0.02,0.98)},anchor=north west},
    width=11cm, height=7cm
]
\addplot[thick, blue, domain=-3:3] {2*x+1};
\addlegendentry{$y=2x+1$ (наклон 2)}
\addplot[thick, red, domain=-3:3] {-0.5*x+1};
\addlegendentry{$y=-\frac12 x+1$ (наклон -0.5)}
\end{axis}
\end{tikzpicture}
\end{center}

\subsection*{1.3 Крива (пример: \(x^2\))}
За \(f(x)=x^2\) наклонът \textbf{не е един и същ навсякъде}.
Точно това „неудобство“ ражда производната.

\subsection*{Въпроси}
\begin{enumerate}
  \item Защо казваме, че „наклонът на \(x^2\) не е един и същ навсякъде“?
  \item Къде очакваш \(x^2\) да е по-стръмна: около \(x=0\) или около \(x=3\)? Защо?
  \item Ако сравниш наклона около \(x=-2\) и около \(x=2\), какъв знак очакваш да имат? (плюс/минус)
\end{enumerate}

\begin{center}
\begin{tikzpicture}
\begin{axis}[
    axis lines=middle, grid=both,
    xmin=-3,xmax=3,ymin=-1,ymax=10,
    xlabel={$x$}, ylabel={$y$},
    width=11cm, height=7cm
]
\addplot[thick, purple, domain=-3:3, samples=200] {x^2};
\end{axis}
\end{tikzpicture}
\end{center}

% =================================================
\section*{2. Големият въпрос}

\begin{quote}
Как да измерим „наклона“ на крива \textbf{в една точка}?
\end{quote}

На права наклонът е константа.
На крива идеята е: първо мерим наклон \textbf{между две точки} (среден наклон), после „сгъваме“
втората точка към първата, докато получим \textbf{наклон в точката}.

\subsection*{Въпроси}
\begin{enumerate}
  \item Каква е разликата между „наклон между две точки“ и „наклон в точка“?
  \item Какво правим с втората точка, за да преминем от секуща към допирателна?
  \item Ако една крива има „ръб“ (ъгъл) в точка, какво подозираш за наклона в тази точка?
\end{enumerate}

% =================================================
\section*{3. Секуща права: наклон между две точки}

\subsection*{3.1 Какво е секуща?}
Вземаме две точки на графиката:
\[
A(x,f(x)),\qquad B(x+h,f(x+h)).
\]
Правата през тях се нарича \textbf{секуща}.

\subsection*{Въпроси}
\begin{enumerate}
  \item Секущата минава през колко точки от графиката?
  \item Ако \(B\) се приближава към \(A\), секущата към какво се стреми?
  \item Ако \(h=0\), можем ли да говорим за секуща по формулата? Защо?
\end{enumerate}

\subsection*{3.2 Наклон на секущата (средна скорост на промяна)}
Наклонът е:
\[
m_{\text{сек}}(h)=\frac{f(x+h)-f(x)}{(x+h)-x}=\frac{f(x+h)-f(x)}{h}.
\]

\textbf{Какво означава това с думи?}
\begin{itemize}
  \item \(f(x+h)-f(x)\) е „колко се е променило \(y\)“.
  \item \(h\) е „колко сме се преместили по \(x\)“.
  \item \(\frac{\Delta y}{\Delta x}\) е „промяна на \(y\) на 1 единица промяна на \(x\)“.
\end{itemize}

\subsection*{Въпроси}
\begin{enumerate}
  \item В израза \(\frac{f(x+h)-f(x)}{h}\) кое е \(\Delta y\) и кое е \(\Delta x\)?
  \item Какво означава „средна скорост на промяна“ с 1 изречение?
  \item Ако \(f(x+h)-f(x)\) е отрицателно и \(h>0\), какъв е знакът на наклона?
\end{enumerate}

\subsection*{Графика: една секуща към \(y=x^2\) при \(x=1\) (само една, да не се претрупва)}
Вземаме \(h=0.5\). Тогава \(A(1,1)\) и \(B(1.5,2.25)\).

\begin{center}
\begin{tikzpicture}
\begin{axis}[
    axis lines=middle, grid=both,
    xmin=-0.2,xmax=2.2, ymin=-0.2,ymax=5.2,
    xlabel={$x$}, ylabel={$y$},
    legend style={at={(0.02,0.98)},anchor=north west},
    width=11cm, height=7cm
]
\addplot[thick, blue, domain=-0.2:2.2, samples=250] {x^2};
\addlegendentry{$y=x^2$}

% точка A(1,1)
\addplot[only marks, mark=*, mark size=2.5pt] coordinates {(1,1)};
\node[above] at (axis cs:1,1) {$A(1,1)$};

% точка B(1.5,2.25)
\addplot[only marks, mark=square*, mark size=2.2pt] coordinates {(1.5,2.25)};
\node[above] at (axis cs:1.5,2.25) {$B(1.5,2.25)$};

% секуща през A и B: наклон 2.5
\addplot[thick, orange, domain=-0.2:2.2] {2.5*x-1.5};
\addlegendentry{секуща при $h=0.5$ (наклон $2.5$)}
\end{axis}
\end{tikzpicture}
\end{center}

\subsection*{3.3 Наблюдение}
Когато \(h\) става по-малко, секущата започва да „залепва“ за графиката около \(A\).
Тя се стреми към една специална права — \textbf{допирателната}.

\subsection*{Въпроси}
\begin{enumerate}
  \item Какво означава „секущата залепва за графиката“ (с думи)?
  \item Коя идея вече сме виждали и в лекция 1: „приближение \(\rightarrow\) граница“ или „факторизация“?
  \item Какво се променя, когато \(h\) става по-малко: точката \(A\), точката \(B\), или и двете?
\end{enumerate}

% =================================================
\section*{4. Допирателна: „най-добрата права“ в точка}

\subsection*{Какво е допирателна?}
Допирателната в точка е права, която локално съвпада по посока с кривата.

\begin{quote}
Мисловен експеримент: ако увеличиш мащаба достатъчно около точка на гладка крива,
кривата започва да изглежда като права. Тази права е допирателната.
\end{quote}

\subsection*{Въпроси}
\begin{enumerate}
  \item Как разпознаваш „добра“ допирателна в една точка (интуитивно)?
  \item Ако увеличаваш мащаба около точка на гладка крива, какво виждаш?
  \item Допирателната пресича ли задължително графиката само в една точка? (Да/не + кратко обяснение.)
\end{enumerate}

% =================================================
\section*{5. Производна: какво \emph{е}}

\subsection*{5.1 Производната като наклон в точка}
\textbf{Производната} \(f'(x)\) е \textbf{наклонът на допирателната} към графиката на \(f\) в точката с абсциса \(x\).

\subsection*{Въпроси}
\begin{enumerate}
  \item Какво е \(f'(x)\) като „геометричен обект“: точка, права, или число?
  \item Ако \(f'(x)>0\), функцията расте ли или спада около тази точка?
  \item Ако \(f'(x)=0\), задължително ли имаме максимум/минимум? (Да/не + 1 изречение.)
\end{enumerate}

\subsection*{5.2 Производната като „моментна скорост“}
Ако \(x\) е време \(t\), а \(f(t)\) е изминато разстояние, тогава:
\[
\frac{f(t+h)-f(t)}{h}
\]
е средната скорост за интервала \([t,t+h]\).
Когато \(h\to 0\), това става \textbf{скоростта в точния момент} \(t\).

\subsection*{Въпроси}
\begin{enumerate}
  \item Каква е разликата между средна и моментна скорост (в езика на формулата)?
  \item Ако \(f(t)\) е температура, какво би означавало \(f'(t)\)?
  \item Защо \(h\to 0\) е важно за „моментната“ идея?
\end{enumerate}

\subsection*{5.3 Защо ни трябва граница?}
Защото за една точка нямаме „втора точка“ за да сметнем наклон.
Правим си втора точка на разстояние \(h\), мерим среден наклон,
и после \textbf{пращаме} \(h\) към 0.

\subsection*{Въпроси}
\begin{enumerate}
  \item Защо не можем просто да сложим \(h=0\) във \(\frac{f(x+h)-f(x)}{h}\)?
  \item Какво правим вместо това? (кажи с 1 изречение)
  \item Какво означава, че „границата съществува“ (интуитивно)?
\end{enumerate}

% =================================================
\section*{6. Пример с \(f(x)=x^2\): наклон в обща точка \(x\)}

Това е ключов момент: ще сметнем наклона „в точката \(x\)“, не само при \(x=1\).

Секущият наклон е:
\[
m_{\text{сек}}(h)=\frac{f(x+h)-f(x)}{h}=\frac{(x+h)^2-x^2}{h}.
\]
Разкриваме \((x+h)^2=x^2+2xh+h^2\):
\[
m_{\text{сек}}(h)=\frac{x^2+2xh+h^2-x^2}{h}=\frac{2xh+h^2}{h}=2x+h.
\]

\textbf{Какво виждаме?}
\begin{itemize}
  \item Докато \(h\neq 0\), това е наклон на секуща: \(2x+h\).
  \item Когато \(h\) стане много малко, \(2x+h\) става много близо до \(2x\).
\end{itemize}

Следователно наклонът „в точката \(x\)“ е \(2x\), тоест:
\[
f'(x)=2x.
\]

\subsection*{Въпроси}
\begin{enumerate}
  \item Как получихме \(2x+h\) от \(\frac{(x+h)^2-x^2}{h}\)? (посочи ключовата стъпка)
  \item Когато \(h\to 0\), кое изчезва и кое остава?
  \item Колко е наклонът на \(x^2\) при \(x=3\) според \(f'(x)=2x\)?
\end{enumerate}

\subsection*{Мини-табличка за усещане}
\[
x=-2 \Rightarrow f'(x)=-4 \ (\text{спада}),\quad
x=0 \Rightarrow f'(x)=0 \ (\text{хоризонтално}),\quad
x=2 \Rightarrow f'(x)=4 \ (\text{расте стръмно}).
\]

% =================================================
\section*{7. Графика: допирателна при \(x=1\) към \(y=x^2\)}

От горното \(f'(1)=2\). Правата с наклон 2 през \((1,1)\) е:
\[
y-1=2(x-1)\quad\Rightarrow\quad y=2x-1.
\]

\subsection*{Въпроси}
\begin{enumerate}
  \item Коя точка използваме, за да построим допирателната при \(x=1\)?
  \item Защо формата \(y-y_0=m(x-x_0)\) е удобна тук?
  \item Ако наклонът беше \(-2\), как визуално щеше да изглежда правата около точката?
\end{enumerate}

\begin{center}
\begin{tikzpicture}
\begin{axis}[
    axis lines=middle, grid=both,
    xmin=-0.5,xmax=2.5, ymin=-1,ymax=6,
    xlabel={$x$}, ylabel={$y$},
    legend style={at={(0.02,0.98)},anchor=north west},
    width=11cm, height=7cm
]
\addplot[thick, blue, domain=-0.5:2.5, samples=250] {x^2};
\addlegendentry{$y=x^2$}

\addplot[thick, red, domain=-0.5:2.5] {2*x-1};
\addlegendentry{допирателна при $x=1$: $y=2x-1$}

\addplot[only marks, mark=*, mark size=2.6pt] coordinates {(1,1)};
\node[above right] at (axis cs:1,1) {$(1,1)$};
\end{axis}
\end{tikzpicture}
\end{center}

% =================================================
\section*{8. Производната като функция}

Производната не е само число. Когато я знаем за всяко \(x\), получаваме нова функция \(f'(x)\).

За \(f(x)=x^2\) получихме \(f'(x)=2x\). Да ги видим заедно:

\subsection*{Въпроси}
\begin{enumerate}
  \item Какво означава „производната е функция“, а не само число?
  \item Къде \(2x\) е положителна, къде отрицателна и какво казва това за \(x^2\)?
  \item Какво означава пресичането на графиката на \(f'(x)\) с оста \(x\)?
\end{enumerate}

\begin{center}
\begin{tikzpicture}
\begin{axis}[
    axis lines=middle, grid=both,
    xmin=-3,xmax=3, ymin=-6,ymax=10,
    xlabel={$x$}, ylabel={$y$},
    legend style={at={(0.02,0.98)},anchor=north west},
    width=11cm, height=7cm
]
\addplot[thick, blue, domain=-3:3, samples=250] {x^2};
\addlegendentry{$f(x)=x^2$}

\addplot[thick, red, domain=-3:3] {2*x};
\addlegendentry{$f'(x)=2x$}
\end{axis}
\end{tikzpicture}
\end{center}

\subsection*{Как да четем картинката?}
\begin{itemize}
  \item Където \(f'(x)>0\) — \(f\) расте.
  \item Където \(f'(x)<0\) — \(f\) спада.
  \item Където \(f'(x)=0\) — често имаме кандидат за минимум/максимум (ще го формализираме по-късно).
\end{itemize}

% =================================================
\section*{9. За какво служи производната (най-полезните 3 приложения)}

\subsection*{9.1 Растене и спадане}
\[
f'(x)>0 \Rightarrow f \text{ расте},\qquad
f'(x)<0 \Rightarrow f \text{ спада}.
\]

\subsection*{Въпроси}
\begin{enumerate}
  \item Ако \(f'(x)<0\) в интервала \((1,4)\), какво прави \(f\) в този интервал?
  \item Ако \(f'(x)=0\) за всички \(x\), каква е функцията \(f\)?
  \item Може ли функция да „расте“ и в същото време производната ѝ да е отрицателна? (Да/не + защо.)
\end{enumerate}

\subsection*{9.2 Минимум и максимум (интуиция)}
Във връх на гладка крива допирателната е хоризонтална, тоест:
\[
f'(x)=0.
\]

\subsection*{Въпроси}
\begin{enumerate}
  \item Защо във връх очакваме \(f'(x)=0\)?
  \item Каква е разликата между „кандидат за екстремум“ и „сигурен екстремум“?
  \item Дай пример за функция, при която \(f'(x)=0\) в точка, но няма максимум/минимум там (само идея).
\end{enumerate}

\subsection*{9.3 Оптимизация (връзка с AI)}
В машинното обучение имаме функция „грешка“ (loss). Искаме да я намалим.
Производната (а по-късно градиентът) казва \textbf{посоката на най-бързото нарастване}.
За да намалим, се движим в обратната посока.

\subsection*{Въпроси}
\begin{enumerate}
  \item Ако „loss“ расте, производната (по параметър) какъв знак има обикновено в посоката на движението?
  \item Защо „отиваме в обратната посока“ на производната при оптимизация?
  \item Какво би означавало „производната е 0“ в контекст на оптимизация?
\end{enumerate}

% =================================================
\section*{10. Официалната дефиниция за производна (в края на интуицията)}

\subsection*{10.1 Дефиниция}
Казваме, че \(f\) е диференцируема в точката \(x\), ако съществува границата:
\[
f'(x)=\lim_{h\to 0}\frac{f(x+h)-f(x)}{h}.
\]
Това число е производната на \(f\) в точката \(x\).

\subsection*{Въпроси}
\begin{enumerate}
  \item Какво е \(h\) в дефиницията — число, точка, или интервал?
  \item Какво измерва \(\frac{f(x+h)-f(x)}{h}\) преди да пуснем \(h\to 0\)?
  \item Защо пишем \(\lim_{h\to 0}\) вместо \(\lim_{h\to \infty}\)?
\end{enumerate}

\subsection*{10.2 Какво ако границата не съществува?}
Тогава казваме, че производната \textbf{не съществува} в тази точка.
Пример (само идея): \(f(x)=|x|\) при \(x=0\) има „ъгъл“ (ляв наклон \(-1\), десен наклон \(+1\)).

\subsection*{Въпроси}
\begin{enumerate}
  \item Какво значи „ляв наклон“ и „десен наклон“?
  \item За \(|x|\) при \(x=0\): какви са тези два наклона?
  \item Ако левият и десният наклон са различни, производната съществува ли?
\end{enumerate}

% =================================================
\section*{11. Правила за производни: извеждане, доказателства и таблица}

\subsection*{11.1 Първи „лесни“ правила (доказуеми директно от дефиницията)}

\textbf{(1) Производна на константа.} Ако \(f(x)=C\), тогава:
\[
f'(x)=\lim_{h\to 0}\frac{C-C}{h}=0.
\]

\textbf{(2) Константен множител.} Ако \(f(x)=C\cdot g(x)\), тогава:
\[
f'(x)=\lim_{h\to 0}\frac{C g(x+h)-C g(x)}{h}
=C\lim_{h\to 0}\frac{g(x+h)-g(x)}{h}
=Cg'(x).
\]

\textbf{(3) Сбор/разлика.} Ако \(f(x)=g(x)+u(x)\), тогава:
\[
f'(x)=\lim_{h\to 0}\frac{[g(x+h)+u(x+h)]-[g(x)+u(x)]}{h}
=\lim_{h\to 0}\left(\frac{g(x+h)-g(x)}{h}+\frac{u(x+h)-u(x)}{h}\right)
=g'(x)+u'(x).
\]

\subsection*{Въпроси}
\begin{enumerate}
  \item Защо производната на константа е 0 (кажи с 1 изречение)?
  \item Ако \(f(x)=7g(x)\), как се променя производната спрямо \(g'(x)\)?
  \item Ако \(f(x)=g(x)-u(x)\), каква е \(f'(x)\)?
\end{enumerate}

\subsection*{11.2 Правило за степен \(x^n\) (идея + доказателство за цели \(n\ge 1\))}
За \(f(x)=x^n\):
\[
f'(x)=\lim_{h\to 0}\frac{(x+h)^n-x^n}{h}.
\]
Разкриваме \((x+h)^n\) по бинома на Нютон:
\[
(x+h)^n=x^n+n x^{n-1}h+\binom{n}{2}x^{n-2}h^2+\cdots+h^n.
\]
Тогава:
\[
(x+h)^n-x^n=n x^{n-1}h+\binom{n}{2}x^{n-2}h^2+\cdots+h^n.
\]
Делим на \(h\):
\[
\frac{(x+h)^n-x^n}{h}=n x^{n-1}+\binom{n}{2}x^{n-2}h+\cdots+h^{n-1}.
\]
Когато \(h\to 0\), всички членове с \(h\) изчезват и остава:
\[
\boxed{\frac{d}{dx}(x^n)=n x^{n-1}}\quad (n=1,2,3,\dots).
\]

\subsection*{Въпроси}
\begin{enumerate}
  \item Каква е производната на \(x^3\) според правилото?
  \item Защо членовете с \(h\) „изчезват“, когато \(h\to 0\)?
  \item Какво става ако \(n=1\)? Получаваме ли правилно, че производната на \(x\) е 1?
\end{enumerate}

\subsection*{11.3 Произведение и частно (ще ги ползваме много)}
\textbf{Правило за произведение:}
\[
(uv)'=u'v+uv'.
\]

\textbf{Кратко доказателство (с дефиницията):}
\[
(uv)'=\lim_{h\to 0}\frac{u(x+h)v(x+h)-u(x)v(x)}{h}.
\]
Добавяме и изваждаме \(u(x+h)v(x)\) в числителя:
\[
=\lim_{h\to 0}\frac{u(x+h)v(x+h)-u(x+h)v(x)+u(x+h)v(x)-u(x)v(x)}{h}.
\]
Групираме:
\[
=\lim_{h\to 0}\left[u(x+h)\frac{v(x+h)-v(x)}{h}+v(x)\frac{u(x+h)-u(x)}{h}\right].
\]
Когато \(h\to 0\), \(u(x+h)\to u(x)\), а дробите \(\to v'(x)\) и \(\to u'(x)\). Значи:
\[
(uv)'=u(x)\,v'(x)+v(x)\,u'(x)=u'v+uv'.
\]

\textbf{Правило за частно:}
\[
\left(\frac{u}{v}\right)'=\frac{u'v-uv'}{v^2}\quad (v\neq 0).
\]
(Може да се получи от правилото за произведение, като \(\frac{u}{v}=u\cdot v^{-1}\).)

\subsection*{11.4 Таблица с основни производни и правила}

\begin{center}
\renewcommand{\arraystretch}{1.35}
\begin{tabular}{|c|c|}
\hline
\textbf{Функция} & \textbf{Производна} \\
\hline
\(C\) & \(0\) \\
\hline
\(x\) & \(1\) \\
\hline
\(x^n\) & \(n x^{n-1}\) \\
\hline
\(\dfrac{1}{x}=x^{-1}\) & \(-\dfrac{1}{x^2}\) \\
\hline
\(\sqrt{x}=x^{1/2}\) & \(\dfrac{1}{2\sqrt{x}}\) \\
\hline
\(e^x\) & \(e^x\) \\
\hline
\(\ln x\) & \(\dfrac{1}{x}\) \\
\hline
\(\sin x\) & \(\cos x\) \\
\hline
\(\cos x\) & \(-\sin x\) \\
\hline
\hline
\textbf{Правило} & \textbf{Формула} \\
\hline
\((u\pm v)'\) & \(u'\pm v'\) \\
\hline
\((Cu)'\) & \(C u'\) \\
\hline
\((uv)'\) & \(u'v+uv'\) \\
\hline
\(\left(\dfrac{u}{v}\right)'\) & \(\dfrac{u'v-uv'}{v^2}\) \\
\hline
\end{tabular}
\end{center}

\subsection*{Въпроси}
\begin{enumerate}
  \item Коя функция има същата производна като себе си?
  \item Каква е производната на \(\cos x\)?
  \item Ако видиш \(\sqrt{x}\), първо би я преписал като \(x^{\alpha}\). Какво е \(\alpha\)?
\end{enumerate}

% =================================================
\section*{12. Задачи (първите 20 от „Домашно 2“) — подробни решения}

\begin{quote}
Съвет за решаване: почти винаги първо \textbf{препиши} функцията като степени, после ползвай правила.
\end{quote}

\subsection*{Въпроси (преди да смятаме)}
\begin{enumerate}
  \item Коя е първата „техническа“ стъпка, която почти винаги правим при производни? (подсказка: степени)
  \item При произведение кое правило ползваме?
  \item При дроб \(u/v\) кое правило ползваме?
\end{enumerate}

% ------------------ 1 ------------------
\subsection*{1) \(\displaystyle y=\frac{1}{x^2}\)}

Преписваме като степен:
\[
y=x^{-2}.
\]
Правило за степен:
\[
y'=-2x^{-3}=-\frac{2}{x^3}.
\]
\[
\boxed{y'=-\frac{2}{x^3}}
\]

% ------------------ 2 ------------------
\subsection*{2) \(\displaystyle y=x\sqrt{x}\)}

Първо като степени:
\[
\sqrt{x}=x^{1/2}\quad\Rightarrow\quad y=x\cdot x^{1/2}=x^{3/2}.
\]
Сега:
\[
y'=\frac{3}{2}x^{1/2}=\frac{3}{2}\sqrt{x}.
\]
\[
\boxed{y'=\frac{3}{2}\sqrt{x}}
\]

% ------------------ 3 ------------------
\subsection*{3) \(\displaystyle y=\frac{4}{x^2}\)}

\[
y=4x^{-2}\quad\Rightarrow\quad y'=4\cdot(-2)x^{-3}=-8x^{-3}=-\frac{8}{x^3}.
\]
\[
\boxed{y'=-\frac{8}{x^3}}
\]

% ------------------ 4 ------------------
\subsection*{4) \(\displaystyle y=\sqrt[3]{x^2}-2\sqrt{x^{-1}}+3x^{-2}-(5x)^{-1}+4\)}

Преписваме всичко като степени:
\[
\sqrt[3]{x^2}=x^{2/3},\qquad \sqrt{x^{-1}}=(x^{-1})^{1/2}=x^{-1/2},
\qquad (5x)^{-1}=\frac{1}{5x}=\frac{1}{5}x^{-1}.
\]
Значи:
\[
y=x^{2/3}-2x^{-1/2}+3x^{-2}-\frac{1}{5}x^{-1}+4.
\]
Диференцираме по членове:
\[
y'=\frac{2}{3}x^{-1/3}-2\cdot\left(-\frac12\right)x^{-3/2}+3\cdot(-2)x^{-3}-\frac15\cdot(-1)x^{-2}+0.
\]
Опростяваме:
\[
y'=\frac{2}{3}x^{-1/3}+x^{-3/2}-6x^{-3}+\frac{1}{5}x^{-2}.
\]
\[
\boxed{y'=\frac{2}{3}x^{-1/3}+x^{-3/2}-6x^{-3}+\frac{1}{5}x^{-2}}
\]

% ------------------ 5 ------------------
\subsection*{5) \(\displaystyle y=x^5\sqrt[3]{x}\)}

\[
\sqrt[3]{x}=x^{1/3}\quad\Rightarrow\quad y=x^5\cdot x^{1/3}=x^{16/3}.
\]
\[
y'=\frac{16}{3}x^{13/3}.
\]
\[
\boxed{y'=\frac{16}{3}x^{13/3}}
\]

% ------------------ 6 ------------------
\subsection*{6) \(\displaystyle y=\frac{5x^2-x-1}{x}\)}

Разделяме поотделно:
\[
y=\frac{5x^2}{x}-\frac{x}{x}-\frac{1}{x}=5x-1-x^{-1}.
\]
\[
y'=5-0-(-1)x^{-2}=5+x^{-2}=5+\frac{1}{x^2}.
\]
\[
\boxed{y'=5+\frac{1}{x^2}}
\]

% ------------------ 7 ------------------
\subsection*{7) \(\displaystyle y=6x^3-4x^2+10x\)}

По членове:
\[
y'=18x^2-8x+10.
\]
\[
\boxed{y'=18x^2-8x+10}
\]

% ------------------ 8 ------------------
\subsection*{8) \(\displaystyle y=(x^2+1)(x^2-x+1)\)}

Ползваме правило за произведение. Нека
\[
u=x^2+1,\quad v=x^2-x+1.
\]
Тогава:
\[
u'=2x,\quad v'=2x-1.
\]
\[
y'=(uv)'=u'v+uv'=(2x)(x^2-x+1)+(x^2+1)(2x-1).
\]
Разкриваме:
\[
(2x)(x^2-x+1)=2x^3-2x^2+2x,
\]
\[
(x^2+1)(2x-1)=2x^3-x^2+2x-1.
\]
Събираме:
\[
y'= (2x^3+2x^3)+(-2x^2-x^2)+(2x+2x)-1
=4x^3-3x^2+4x-1.
\]
\[
\boxed{y'=4x^3-3x^2+4x-1}
\]

% ------------------ 9 ------------------
\subsection*{9) \(\displaystyle y=(1-3x+7x^2)(-5x^2-1)\)}

Пак произведение. Нека:
\[
u=1-3x+7x^2,\quad v=-5x^2-1.
\]
Тогава:
\[
u'=-3+14x,\quad v'=-10x.
\]
\[
y'=u'v+uv'=(-3+14x)(-5x^2-1)+(1-3x+7x^2)(-10x).
\]
Разкриваме първия член:
\[
(-3+14x)(-5x^2-1)=15x^2+3-70x^3-14x.
\]
Вторият:
\[
(1-3x+7x^2)(-10x)=-10x+30x^2-70x^3.
\]
Събираме:
\[
y'=(-70x^3-70x^3)+(15x^2+30x^2)+(-14x-10x)+3
=-140x^3+45x^2-24x+3.
\]
\[
\boxed{y'=-140x^3+45x^2-24x+3}
\]

% ------------------ 10 ------------------
\subsection*{10) \(\displaystyle y=x^2(1-x^2)\)}

Първо опростяваме:
\[
y=x^2-x^4.
\]
\[
y'=2x-4x^3.
\]
\[
\boxed{y'=2x-4x^3}
\]

% ------------------ 11 ------------------
\subsection*{11) \(\displaystyle y=\frac{1-x}{1+x}\)}

Частно. Нека \(u=1-x\), \(v=1+x\).
\[
u'=-1,\quad v'=1.
\]
\[
y'=\frac{u'v-uv'}{v^2}=\frac{(-1)(1+x)-(1-x)(1)}{(1+x)^2}.
\]
Числител:
\[
-(1+x)-(1-x)=-2.
\]
\[
\boxed{y'=-\frac{2}{(1+x)^2}}
\]

% ------------------ 12 ------------------
\subsection*{12) \(\displaystyle y=\frac{x}{x^2-1}\)}

\(u=x\Rightarrow u'=1\), \(v=x^2-1\Rightarrow v'=2x\).
\[
y'=\frac{u'v-uv'}{v^2}
=\frac{(x^2-1)-2x^2}{(x^2-1)^2}
=-\frac{x^2+1}{(x^2-1)^2}.
\]
\[
\boxed{y'=-\frac{x^2+1}{(x^2-1)^2}}
\]

% ------------------ 13 ------------------
\subsection*{13) \(\displaystyle y=\frac{5}{6x-4}\)}

\[
y=5(6x-4)^{-1}.
\]
\[
y'=5\cdot\left[-6(6x-4)^{-2}\right]=-30(6x-4)^{-2}.
\]
\[
\boxed{y'=-\frac{30}{(6x-4)^2}}
\]

% ------------------ 14 ------------------
\subsection*{14) \(\displaystyle y=\frac{2x-3}{5-4x}\)}

Частно: \(u=2x-3\Rightarrow u'=2\), \(v=5-4x\Rightarrow v'=-4\).
\[
y'=\frac{2(5-4x)-(2x-3)(-4)}{(5-4x)^2}.
\]
Числител:
\[
2(5-4x)=10-8x,\qquad (2x-3)(-4)=-8x+12,
\]
\[
10-8x-(-8x+12)=-2.
\]
\[
\boxed{y'=-\frac{2}{(5-4x)^2}}
\]

% ------------------ 15 ------------------
\subsection*{15) \(\displaystyle y=\frac{2x^2}{1-7x}\)}

Частно: \(u=2x^2\Rightarrow u'=4x\), \(v=1-7x\Rightarrow v'=-7\).
\[
y'=\frac{4x(1-7x)-2x^2(-7)}{(1-7x)^2}
=\frac{4x-28x^2+14x^2}{(1-7x)^2}
=\frac{4x-14x^2}{(1-7x)^2}.
\]
\[
4x-14x^2=2x(2-7x).
\]
\[
\boxed{y'=\frac{2x(2-7x)}{(1-7x)^2}}
\]

% ------------------ 16 ------------------
\subsection*{16) \(\displaystyle y=\frac{3}{6-x}\)}

\[
y=3(6-x)^{-1}.
\]
\[
\frac{d}{dx}(6-x)^{-1}=-(6-x)^{-2}\cdot(-1)=(6-x)^{-2}.
\]
\[
y'=3(6-x)^{-2}=\frac{3}{(6-x)^2}.
\]
\[
\boxed{y'=\frac{3}{(6-x)^2}}
\]

% ------------------ 17 ------------------
\subsection*{17) \(\displaystyle y=\frac{-7}{3-10x}\)}

\[
y=-7(3-10x)^{-1}.
\]
\[
\frac{d}{dx}(3-10x)^{-1}=-(3-10x)^{-2}\cdot(-10)=10(3-10x)^{-2}.
\]
\[
y'=-7\cdot 10(3-10x)^{-2}=-70(3-10x)^{-2}.
\]
\[
\boxed{y'=-\frac{70}{(3-10x)^2}}
\]

% ------------------ 18 ------------------
\subsection*{18) \(\displaystyle y=\frac{ax+b}{cx+d}\)}

\[
u=ax+b\Rightarrow u'=a,\qquad v=cx+d\Rightarrow v'=c.
\]
\[
y'=\frac{a(cx+d)-(ax+b)c}{(cx+d)^2}.
\]
\[
a(cx+d)-(ax+b)c=acx+ad-acx-bc=ad-bc.
\]
\[
\boxed{y'=\frac{ad-bc}{(cx+d)^2}}
\]

% ------------------ 19 ------------------
\subsection*{19) \(\displaystyle y=\sin x\cos x\)}

\[
y'=\cos x\cos x+\sin x(-\sin x)=\cos^2 x-\sin^2 x=\cos 2x.
\]
\[
\boxed{y'=\cos 2x}
\]

% ------------------ 20 ------------------
\subsection*{20) \(\displaystyle y=x^3\cos x\)}

\[
u=x^3\Rightarrow u'=3x^2,\qquad v=\cos x\Rightarrow v'=-\sin x.
\]
\[
y'=3x^2\cos x-x^3\sin x.
\]
\[
\boxed{y'=3x^2\cos x-x^3\sin x}
\]

% =================================================
\section*{13. Какво следва (Лекция 3)}

Следващия път:
\begin{itemize}
  \item верижно правило (chain rule) + производни на „сложни“ функции
  \item повече задачи за допирателни, производна в точка, графични интерпретации
  \item първи задачи за екстремуми и оптимизация (вкл. мини-паралел с gradient descent)
\end{itemize}

\end{document}
